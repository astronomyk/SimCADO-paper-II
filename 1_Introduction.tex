\section{Introduction}
\label{sec:introduction}

\begin{table*}

    \centering
    \caption{A compilation of mass limits for a selection of studies of the IMF outside the Milky Way with the Hubble space telescope. It should be noted that for the study by \citet{gallart1999} the estimated global star formation history was consistent with a Salpeter slope, rather than a Salpeter slope being extracted from the photometric data.}
    \label{tbl:imf_lit_review}

    \begin{tabular}{ l l r r r r r }

        \hline
        \hline
        Galaxy   &  Target      &  Distance &  Mass range       & IMF Slope(s) & Break Mass          & Reference         \\
                &               & kpc       & \msun             &              & \msun               &                   \\
        \hline                  
        LMC      &  R136        & 50        & 2.8-15            & 2.22         &                     & Hunter 1995       \\
        LMC      &  NGC 1818    & 50        & 0.85-9            & 2.23         &                     & Hunter 1997       \\
        LMC      &  R136        & 50        & 1.35-6.5          & 2.28, 1.27   & 2.1                 & Sirianni 2000     \\
        LMC      &  LH 95       & 50        & 0.43-20           & 2.05, 1.05   & 1.1                 & Da Rio 2009       \\
        \hline                                                  
        SMC      &  NGC 330     & 62        & 1-7               & 2.3          &                     & Sirianni 2002     \\
        SMC      &  NGC 602     & 62        & 1-45              & 2.2          &                     & Schmalzl 2008     \\
        SMC      &              & 62        & 0.37–0.93         & 1.9          &                     & Kalirai 2013      \\
        \hline                                                  
        Hercules &              & 135       & 0.52-0.78         & 1.2          &                     & Geha 2013         \\
        Leo IV   &              & 156       & 0.54-0.77         & 1.3          &                     & Geha 2013         \\
        Leo I*   &              & 250       & 0.6-30            & 2.3          &                     & Gallart 1999      \\  
        \hline
        \end{tabular}

\end{table*}


The stellar Initial Mass Function (IMF), or the spectrum of stellar masses at birth, has implications in almost all fields of astrophysics. On the local scale, the IMF determines the number of available massive stars and with it the fate of a star formation region and creating the environments that emergent planet-forming circumstellar disks will be exposed too. 
On the large scale, the IMF is irrevocably connected to the composition of the stellar populations in a galaxy and has a critical impact on the mass and energy cycle of a galaxy. The larger the amount of mass locked up in low mass stars, the smaller the reservoir of gas available for the next generation of stars and consequently the smaller the potential for the enrichment of the interstellar medium (ISM). Finally, cosmological simulations of galaxy formation and large scale structure inevitably rely on a universal IMF to determine stellar yields and the strength of feedback mechanisms governing the transport of energy and material. 
\footnote{\url{https://simcado.readthedocs.io/}}
%In short, the IMF is a fundamental parameter in astronomy. 

% In his original work, \citet{salpeter1955} used a single power law distribution with a slope of 2.35 to describe the IMF for masses greater than \s1\,\msun. This was later modified to a series of broken power laws to include the stars below the hydrogen burning limit by \citet{kroupa2001}. In contrast, \citet{chabrier2003} proposed a log-normal distribution with a power law modification for the high and low mass regions. As the two descriptions are very similar in the most populated region between 0.1\msun and 10\msune, it has proved  difficult to decide which model more aptly describes the IMF.

In his original work, \citet{salpeter1955} used a single power-law distribution with a slope of 2.35 to describe the IMF for masses between \s1\ and \s10\,\msun . This description was later modified to a series of broken power laws to include the stars below the hydrogen-burning limit by \citet{kroupa2001}. \citet{chabrier2005} proposed a log-normal distribution with a power-law modification for the high mass regions. As both descriptions are empirical and not the prediction from theory, it has proved difficult from observations to decide which of these two descriptions more aptly describes the IMF.

Most observational studies suggest that the shape of the IMF is constant \citep{Lada2003-ip,Kroupa2002,Bastian2010}. Definitive deviations from the accepted IMF form are elusive, and when found, often controversial \citep{Van_Dokkum2010-gx,Conroy2012-hv,Drass2016-kp}. One major challenge hinders a concluding result on the universality of the IMF:  directly derived IMFs from star-counts for environments substantially different from the solar neighbourhood. 
Table \ref{tbl:imf_lit_review} shows that even in the closest star forming galaxies like the Magellanic clouds, only the Hubble space telescope (HST) has the sensitivity to reach below one solar mass (see references in Table \ref{tbl:imf_lit_review}). 
Long exposures with HST have observed stars just below the first break in the Kroupa power law at 0.5\,\msun \citep{dario2009,kalirai2013,geha2013}, but not far enough into the lower mass regions to put reliable constraints on the shape of the IMF in these extragalactic environments. 
Adding to observers' woes is the lack of spatial resolution. At the distance of the LMC, star forming regions can contain anywhere from 10\h2 to 10\h5\,\spae. 
Figure~1 of \citet{sirianni2000} shows a perfect example of why current studies struggle to reliably determine the IMF for dense stellar populations outside the Milky Way. The depicted cluster core (R136) is completely dominated by the flux of a few of the brightest stars.
Thus studies of the IMF are limited to the outer regions of these clusters where stellar densities are low enough for individual low mass stars to be resolved. Unconstrained mass segregation can also skew the results when considering the IMF in massive clusters (e.g., \citealt{Ascenso2009-de}). More importantly, without being able to study the massive clusters in- and outside the Milky Way, it is difficult to make strong assertions on the universality of the IMF. In order to systematically study unambiguously (via star-counts) the lower mass part of the IMF and be able to characterize differences between IMFs, telescopes with higher spatial resolution and better sensitivity than the current generation of ground and space based telescope is needed.

In the middle of the next decade, the era of the extremely large telescopes will begin. 
ESO's Extremely Large Telescope (ELT) \citep{eelt} with the help of advanced adaptive optics \citep{maory} will have the power to resolve spatial scales at the diffraction limit of a 40m-class mirror. This will provide a linear improvement of a factor of \s15$\times$ over HST and a factor of \s6$\times$ over the future JWST telescope. 
With a collecting area of 978\,m\h2 the ELT will have at least the same sensitivity as the HST in sparse field and will be able to observe much deeper than HST in crowded fields. 
The MICADO instrument \citep{micado2016, micado2018} will be the ELT's first-light near-infrared (NIR) wide-field imager and long slit spectrograph. 
With a diffraction limit of 7\,mas at 1.2\um and an AO corrected field of view of almost a square arcminute, MICADO will be perfectly suited to address exactly the IMF science case. 

The main focus of this paper is to determine to what extent MICADO will improve our ability to study the IMF and other properties of dense stellar populations. More precisely, we have attempted to addressed the following two questions: 
1) What is the lowest mass star that MICADO will be able to observe for a given density and distance? and
2) What instrumental effects will play a critical role when undertaking such studies with MICADO and the ELT?
%How crowded can a region be for MICADO to still be able to detect 90\% of the stars above the detection limit? 
In our quest for answers we used SimCADO, the instrument data simulator for MICADO \citep{leschinski2016}, to simulate a wide range of densely populated stellar fields at various distances. 
The current version of SimCADO takes into account all the major and most of the minor spatial and spectral effects along the line of sight between the source and the detector. 
We use the software to generate realistic images of model stellar fields and  conduct several iterations of PSF photometry and star subtraction to extract as many stars as possible from the simulated observations. 
The extracted stars were compared with the input catalogue to determine the completeness of the extraction and to define a ``limiting reliably observable mass'' for the different stellar field densities and distances.

This paper is organised in the following way: Section \ref{sec:observations} describes the stellar fields used in our simulations, how the simulations were run, and also describe the algorithm for detecting and subtracting stars in the simulated images. 
In Section \ref{sec:results} we describe the results of the simulations and discuss their validity in the context of possible future observations of real young stellar clusters. 
Section \ref{sec:conclusion} summarises our results.
