\section{Conclusion}
\label{sec:conclusion}

MICADO and the ELT will provide the chance to finally resolve the core populations of the densest star clusters in the Milky Way and in neighbouring galaxies out to distances of a several hundred kiloparsecs. By turning MICADO towards young stellar clusters we hope to finally answer the question as to whether the IMF distribution is indeed universal, or whether its shape changes when we leave the solar neighbourhood. Currently these answers are locked up inside the dense cores of young stellar clusters. Observations of these clusters are primarily limited by confusion. Determining exactly how much more of these stellar populations will be visible to MICADO was the primary goal of this work.

This study aimed to answer two questions: What is the lowest mass star that MICADO will be able to observe reliably for a given stellar density and distance? And what instrumental effects will play a critical role when undertaking such studies with MICADO and the ELT? In order to answer these we used the instrument simulator for MICADO (SimCADO) to generate ``observations'' of 42 dense stellar regions corresponding to the cores of young stellar clusters at varying distances from earth. Here we present a brief summary of the results:

\rewrite HOW BEST TO ORDER THE CONCLUSIONS?

\begin{itemize}
    \item We have shown that MICADO will easily be able to resolve all members of a stellar population with a density up to 10\h3\,\spa. With proper knowledge of the PSF and an optimised detection and subtraction algorithm densities of 5$\times$10\h3\,\spa should also be achievable.
    
    \item Observations with MICADO will enable direct (resolved) observations of the IMF in well over 1500 young clusters in diverse environments within the local group of galaxies. This will provide a statistically meaningful sample for robustly quantifying the role of the environmental on the shape of the IMF.

    \item Given that MICADO observations will be constrained by the instrument's sensitivity to both the brightest and faintest sources in the field of view, MICADO is best suited to investigate the shape of the IMF in clusters in the outer edges of the Milky Way as well as the Magellanic Clouds. For example, MICADO is perfectly suited to resolve the IMF in cores of dense young star clusters such as R136 in the LMC and NGC330 in the SMC.
    
    \item Observations focusing on the initial mass function of clusters in the LMC will be limited by sensitivity, not crowding, to 0.1\,\msun. Investigations of the brown dwarf knee (\s0.08\,\msun) will not be possible outside the Milky Way, however MICADO's resolution will allow the peak of the IMF (0.1\,\msun\textless M \textless0.5\,\msune) to be extensively investigated in the Magellanic clouds, and the Salpeter slope of the high mass region of IMF out to distances of 5\,Mpc.
    
    \item The brown dwarf population will be accessible in the cores of the densest Milky Way clusters, e.g. in the Arches and Westerlund clusters. Objects with masses on the order of 10\,M$_{Jup}$ will be accessible by MICADO for clusters within 8\,kpc of Earth. The only caveat is that an appropriate observation strategy must be found to mask the many bright (m$_{Ks}<15^m$) stars present in all Milky Way clusters.
    
    \item Finally accurate knowledge of the ELT's PSF will be absolutely essential for good photometry and PSF subtraction algorithms. The sharp structures created by the segmented mirror design will lead to many fake low luminosity star detections if either the PSF is not well known or the extraction algorithm is not capable of differentiating between a star and an artifact of the PSF.
    
\end{itemize}


% The parameter space covered distances from 8\,kpc to 5\,Mpc and stellar densities from 10\h2\,\spa to \s10\h5\,\spa - densities much higher then either Hubble or JWST will be capable of resolving. 


% - Answer the questions in the introduction

    % - At distance X and for stellar density Y, down to which mass (Z) can I get an IMF?
    
    % - If variations exist in the IMF, what level could MICADO detect at which distance?
    
% - Brief comparison to JWST's capabilities and where MICADO will accel

% - Brief note on how important the need for PSF-Reconstruction is and how we need to develop tools for source detection and extraction which can properly deal with non-smooth PSFs