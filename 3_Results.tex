\section{Results and Discussion}
\label{sec:results}

\subsection{The lowest reliably observable masses for given stellar densities and distances.}

\begin{figure*}

    \centering
    \includegraphics[width=\textwidth]{images/old_trusted_mass}

    \caption{This graph is the answer to the first question we posed: what is
    the lowest observable mass for given stellar densities and distances? The
    errors in the observable mass are 0.2 dex and correspond to the size of the
    mass bins used. Two trends are visible in the best fit lines for each
    distance: the flat regime shows that the limiting mass is based on the
    sensitivity limit of MICADO, while the exponential regime shows where
    crowding becomes the limiting factor. The cavity to the lower right shows
    the parameter space in which stars of a given mass will not be observable.
    Here the stellar density includes all the stars in a given area down to
    0.01\,\msun, not just the stars above the sensitivity limit. Hence for the
    cases where observations are sensitivity limited, the effective observable
    star density is, in the cases with greater distance, much lower. In these
    cases the stars below the sensitivity limit only contribute to a higher
    background flux. }
    
    \label{fig:trusted_mass}
    
\end{figure*}

\begin{figure*}

    \centering
    \includegraphics[width=\textwidth]{images/star_density_vs_age}

    \caption{The stellar densities in the cores of the clusters listed in  Table
     A\ref{tbl:pz10_selection} assuming a sensitivity limit of K$_S$=28\m.  The
     size of the circles is proportional (including an offset) to the relative
     on-sky size of the cluster cores. The colours reflect the lowest possible
     reliably observable mass, as shown in Figure \ref{fig:trusted_mass} and
     listed in Table A\ref{tbl:pz10_selection}. Brown: M\textgreater0.01\msun;
     Pink: M\textgreater0.1\msun; Yellow: M\textgreater0.9\msun. The densities
     shown here take into account the sensitivity limit and therefore are only
     for the potentially observable stars, i.e. any low luminosity stars with
     K$_S$\textgreater28\m are omitted from the density calculation. This is
     equivalent to all stars in the Milky Way, M-class stars and brighter in
     the LMC, and G-class stars and brighter in M31. Only clusters from
     \citet{portegies2010} which have a defined core radius, r$_c$, are shown.
     The dashed lines in this figure represent the limit to the resolving
     capability of the HST, JWST and ELT. We define the limiting density as
     the mean distance between stars being equal to 2.5$\times$ the PSF FWHM.
     Given the predicted PSF shapes for the latter two telescopes, these lines
     may prove to be somewhat optimistic. Nevertheless the graphic
     illustrates the point that cores of the majority of young clusters are
     far too dense for either HST or JWST observations. Thus it will require
     the ELT (or similar) to study the most heavily populated regions of
     these clusters.}
    
    \label{fig:star_density_vs_age}
    
\end{figure*}



The first of the questions we asked with this study -- ``What is the lowest
mass star that MICADO will be able to observe reliably for a given density and
distance?'' -- can be answered by Figure \ref{fig:trusted_mass}. For each of
the distances and densities we have plotted the lowest reliable mass bin. The
scatter in the plot reflects the random nature of the simulations. The
positions of the stars in each of the stellar field were randomised, the
sampling of the mass function was random and detector and shot noise was
applied to the image as part of SimCADO's read-out process. Thus no two  stellar
fields were the same. Each stellar field configuration was only run once. We
therefore only have one data point for each density and distance. The bin size
used for the reliability statistics was set to 0.2 dex, and is therefore the
uncertainty in the limiting observable mass.

From Figure \ref{fig:trusted_mass} we can immediately see the two limiting
regimes of sensitivity and crowding. The flat parts of the curves in Figure
\ref{fig:trusted_mass} show the densities for which MICADO will be sensitivity
limited at each distance and the diagonal regions show when crowding becomes
the limiting factor. For example observations of a cluster at a distance of 8\,
kpc observations will always be crowding limited for densities above 100\,\spae.
At a distance of 200\,kpc observations will be limited by sensitivity up to a
density of 10\h4\,\spa, thereafter crowding will be the dominant factor. At
5\, Mpc all observations will be sensitivity limited. As a reference we have
included the approximate stellar densities for three well known young  clusters
in Figure \ref{fig:trusted_mass} \textit{if they were located at the distance
of the simulated clusters}. For example, if the YMC Westerlund 1 were to be
located in the LMC, it would fall in to the crowding-limited regime for MICADO.
The lowest reliably observable mass in the densest region of the core would
only be \s0.5\,\msun. This is equivalent to what HST is capable of observing
in the outer rim territories of LMC clusters. For clusters in the LMC with
stellar densities less than 10\h3 MICADO will be limited by sensitivity to
masses above 0.1\,\msun. While this mass is only 0.3\,\msun lower than what
current observations with Hubble can achieve, it should be emphasised that
this increase of ``only'' 0.3\,\msun will reveal the majority of M-type stars,
which account for almost three quarters of all main sequence stars
\citep{ledrew2001}. Given that the limit of current studies is around
the 0.5\,\msun knee from \citet{kroupa2001}, opening up this range will allow
future studies to pin down exactly what the shape of the IMF looks like in the
LMC clusters.

As previously noted the exposure time for the simulated images was one hour. By
observing for longer times, the lowest observable mass will decrease, however
the change is disproportionate to the exposure time. \citet{leschinski2016}
show that increasing the exposure time to 10 hours per cluster only increases
the sensitivity limit by around 1.5\m and 1\m in the J and K$_S$ filters
respectively. For the case of the LMC, this would decrease the lowest
observable mass to around 0.06\msune, i.e. just below the hydrogen burning
limit.

It should also be noted the majority of young clusters have cores which are
less dense than that of Westerlund 1, and therefore the limiting observable
mass will also be lower than the 0.5\,\msun mass quoted for a Westerlund 1-like
YMC in the LMC. Given MICADO's resolving power it will therefore also be
possible to determine to what extent apparent mass segregation has played a
role in previous studies of the IMF in the LMC. More to the point MICADO will
enable us to understand the apparent deviations from the Salpeter IMF as
reported by \citet{dario2009}, \citet{geha2013} and \citet{kalirai2013}.

At distances of 100\,kpc to 200\,kpc and with careful photometry and longer
observations MICADO should be able to detect stars down to the sensitivity
limit of 0.5\,\msun. This will only be possible though for clusters with
stellar densities less than 10\h4 \spa. As a reference an ONC-like cluster at a
distance of 200\,kpc will have a stellar density on the order of 10\h5 \spa.
Such observations will be useful for determining the composition of OB
associations and sparser (older) open clusters, if there were any present in
the non-Magellanic satellites of the Milky Way. Nevertheless MICADO will still
allow us observe the fabled 0.5\,\msun knee in the field population of the
nearest low metallicity dwarf spheroidal galaxies.

Closer to home MICADO should be able detect 10\,M$_{Jup}$ objects in an
ONC-like clusters at a distance of 8\,kpc. The Arches cluster is an obvious
candidate for such studies, and given its proximity to the galactic centre
makes it an ideal case to study the IMF under extreme conditions. The main
hindrance to such observations is not the almost 2 mag of variable Ks-band
extinction  along the line of site \citep{espinoza2009}, but rather the
brightest stars in the cluster. The effectiveness of MICADO observations will
also be limited by the brightest stars in the field. Leschinski (2018, in prep)
state that point sources with magnitudes K$_S$\textgreater14.8\,\m will
saturate the MICADO detectors within the 2.6\,s minimum exposure time. There
are very few regions in the cores of Milky Way open clusters which do not
contain stars brighter than K$_S$\s15\m, making deep MICADO observations of
these regions difficult.


\subsection{The cores densities of young star clusters}

The second of the questions we asked with this study was ``What instrumental
effects will play a critical role when undertaking such studies with MICADO and
the ELT?''. The instrumental effect which would play the largest role
regarding the accuracy of the estimates given here is our knowledge of the PSF.
For this study we used a single SCAO PSF. We assumed that the PSF orientation
stayed the same for the length of the observation. Consequently we had a very
good model of our reference star for the PSF subtraction. This will obviously
not be the case for real observations as the pupil of the telescope will rotate
with respect to the sky, causing an axial broadening of the PSF over the course
of an observing run. On the one hand this broadening should improve the results
from our subtraction method as it will smooth out many of the sharp features
of the instantaneous PSF. On the other hand we will lose information on both
the structure of the PSF and the extent of the wings. Thus the PSF subtraction
algorithm will less accurately be able to estimate the background level when
fitting the reference PSF to a star. As a consequence faint stars caught in the
PSF wings of the brighter stars may not be detected as often as they would be
if the PSF remained rotationally aligned with the sky. A hybrid approach to the
faint star subtraction problem may be the following: Subtract the brightest
stars from each individual exposure using an instantaneous PSF derived from the
brightest stars in that exposure, then stack the residual images and extract
the faintest stars using a rotationally broadened PSF. Further investigation is
required to determine whether this approach would indeed increase the detection
rate for faint stars.

Although it may seem obvious, one final point is worth mentioning. From our
simulations it is clear that resolving stellar densities of 10\h3\,\spa is well
within the capabilities of MICADO. With an optimised PSF fitting and
subtraction algorithm, extracting upwards of 5$\times$10\h3\,\spa should also
be in the realms of possibility. 5$\times$10\h3\,\spa is equivalent to
approximately one star in the equivalent area of \s2.5 ELT H-band PSF FWHMs.
This is similar to being able to resolve every star in the core of an ONC-like
cluster in the LMC. For JWST and HST the equivalent stellar densities are only
160~\spa and 20~\spa respectively. Although MICADO may not have the sensitivity
of a space-based telescope, the resolving power will give us full access to the
core populations of dense stellar clusters in the major satellites of the
Milky Way.


\subsection{The cores densities of young star clusters}

These simulations are a nice theoretical exercise, however without an
application to observations they are not all that useful.
Figure~\ref{fig:star_density_vs_age} shows the estimated stellar densities in
the cores of the open clusters and YMCs compiled by \citet{portegies2010}. The
density values, log$_{10}$($\rho$), only take into account the stars with
apparent magnitudes above the sensitivity limit of MICADO and thus reflect the
``real'' observable density for the clusters (Also listed in Table
A\ref{tbl:pz10_selection}). The limits set for HST, JWST and MICADO are the
critical stellar density above which our extraction algorithm struggles to
detect and remove more than 90\% of the stars in a field. We find that for the
Galactic clusters, the resolution of JWST will be sufficient to resolve all
stars in the cluster's core down to the sensitivity limit of the instrument.
For clusters in the galactic plane though JWST observations will struggle to
disentangle the cluster stars from the field stars. To robustly determine
cluster membership observations of the proper motion of the cluster relative to
the field will be required. \citet{stolte2008} show that the proper motion of
the Arches cluster near the Galactic centre is \s5\,mas yr\h{-1}, around a
sixth the size of a pixel in the JWST NIRCam instrument. MICADO, in contrast,
will have a plate scale of 1.5\,mas in the zoom mode, meaning the cluster's
members could be determined by observations spaced only several months apart.

Resolving the cores of the young clusters in the Magellanic clouds will not be
possible with JWST. Based on the compiled ages listed in \citet{portegies2010}
MICADO should give us access to the cores of young clusters in the LMC which
cover a wide range of ages. This will allow a much deeper understanding of the
dynamical processes (e.g. evaporation, core collapse, etc.) involved in the
evolution of these clusters. Additionally observations of a series of LMC
clusters with varying ages will give a much better picture of how the initial
mass function evolves into the present day mass function, and how the dynamical
evolution of the cluster influences the observations and calculations of a
cluster's IMF.


\subsection{The cores densities of young star clusters}

\begin{figure*}

    \centering
    \includegraphics[width=\textwidth]{images/young_clusters_within_2Mpc}

    \caption{The blue lines show the cumulative number of young clusters with
    distance from Earth as reported in catalogues and the literature. The red
    line shows the

    that we expect to be able to see with MICADO. The blue line is the
    cumulative sum of clusters from the literature. The red line is the
    projected number of clusters that can be inferred to exist based on the
    total H$\_alpha$ flux of the galaxies. The conversion is}

    The dotted grey line shows a Kroupa broken power law IMF.


    \label{fig:resolved_stellar_densities}

\end{figure*}





% \begin{figure*}

    % \centering
    % \includegraphics[width=\textwidth]{images/star_density_motivation}

    % \caption{}
    
    % \label{fig:resolved_stellar_densities}
    
% \end{figure*}


% What is the lowest mass star that MICADO will be able to observe for a given density and distance? 
% How crowded can a region be for MICADO to still be able to detect 90% of the stars above the detection limit? 
% What instrumental effects will play a critical role when undertaking such studies with MICADO and the ELT?


% Results}
% - 
% * ELT PSFs make life very difficult - many fake sources
% * MICADO can get well into the Brown dwarf regime (>0.01Msun) for all clusters within 8kpc of the Sun
% * It will just miss out on the BD knee in the LMC (>0.1Msun). But it will catch the true structure of the 0.5Msun knee.
% * Densities of ~1000 stars/arcsec2 are easily resolvable. 5000 stars/arcsec2 if we have good knowledge of the PSF.
% * This is equivalent to every star in the Arches cluster at a distance of the LMC. ONC would be fully resolvable (assuming no extinction) in Leo I Dwarf (220kpc)
% * in the LMC, meaning MICADO can basically resolve out all stars in every YMC listen in Portegeis-Zwart 2010



% NEED COMPLETENESS NUMBERS - HOW MANY ACTUALLY CAME BACK!!!!

    % - At what densities does the PSF subraction break down?

    % - At what desntites do we start getting fake sources from the PSF artifacts?

% - What is the limiting mass vs distance vs stellar density

% - What are the effective densities for the cases where the limiting mass is resolution limited?
% - Where does JWST drop out?

    % - Examples of the clusters (i.e. ONC in NGC300) where MICADO will make the biggest contribution
    
% - What level of variation in each mass regime (M<0.08, 0.08<M<0.5, M>0.5) do we see?
