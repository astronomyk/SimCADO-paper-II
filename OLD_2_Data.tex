\section{Data sets and Simulations}

\subsection{What parameter space are we covering?}
    - Why did we choose this parameter space?

\subsection{How did we generate the data set?}

    - How did we generate the clusters?

    - What did the optical train look like?

    - What observing time did we use?

    - What assumptions and biases maybe in the data?

    - WHat level of IMF variation did we build into the data sets

\subsection{How did we extract the sources and fit the data?}

    - Describe the source finding and PSF extraction package/scripts

    - At what densities does the PSF subraction break down?

    - At what desntites do we start getting fake sources from the PSF artifacts?



Range of parameters:

stellar densities
- 100 stars/arcsec2 --> 1E6 stars/arcsec2

Distances
- 8kpc, 50kpc, 200kpc, 800kpc, 2Mpc, 5Mpc


Field of View : 2" x 2"
PSF : MAORY SCAO
Exposure time : 1 hour
Filter : Ks
Cluster Age : 50 Myr (** should re-run for 5 Myr)

IMF Variations :
Cluster Mass : 1000

For each section (M<0.08, 0.08<M<0.5, M>0.5), use standard alphas (0.3, 1.3, 2.3) and do +/- (0.1, 0.5)
